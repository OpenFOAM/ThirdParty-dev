%%%%%%%%%%%%%%%%%%%%%%%%%%%%%%%%%%%%
%                                  %
% Titre  : p_c.tex                 %
% Sujet  : Manuel de l'utilisateur %
%          du projet 'PT-Scotch'   %
%          Changes                 %
% Auteur : Francois Pellegrini     %
%                                  %
%%%%%%%%%%%%%%%%%%%%%%%%%%%%%%%%%%%%

\section{Updates}
\label{sec-changes}

\subsection{Changes from version 5.0}

\ptscotch\ now provides routines to compute in
parallel partitions of distributed graphs.

A new integer index type has been created in the Fortran interface, to
address array indices larger than the maximum value which can be
stored in a regular integer. Please refer to
Section~\ref{sec-install-inttypesize} for more information.

A new set of routines has been designed, to ease the use of the
\libscotch\ as a dynamic library. The {\tt SCOTCH\_\lbt version}
routine returns the version, release and patchlevel numbers of the
library being used. The {\tt SCOTCH\_\lbt *Alloc} routines,
which are only available in the C interface at the time being,
dynamically allocate storage space for the opaque API
\scotch\ structures, which frees application programs from the need
to be systematically recompiled because of possible changes of
\scotch\ structure sizes.

\subsection{Changes from version 5.1}

Unlike its sequential counterpart, version {\sc 6.0} of
\ptscotch\ does not bring major algorithmic improvements with respect
to the latest {\sc 5.1.12} release of the {\sc 5.1} branch.

In order to ease the work of people writing numerical solvers, it
exposes in its interface a new distributed graph handling routine,
{\tt SCOTCH\_\lbt dgraph\lbt Redist}, that builds a redistributed
graph from an existing distributed graph and partition data. See
Section~\ref{sec-lib-dgraphredist}.
